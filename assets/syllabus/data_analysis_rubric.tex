\documentclass{article}

\usepackage{amssymb,amsmath}
% \usepackage{mathspec}
\usepackage{graphicx,grffile}
\usepackage{longtable}
\usepackage{booktabs}

\usepackage{geometry}
\geometry{landscape, margin=0.5in}

\newtheorem{mydef}{Definition}
\newtheorem{thm}{Theorem}
\DeclareMathOperator*{\argmin}{arg\,min}
\DeclareMathOperator*{\argmax}{arg\,max}

\providecommand{\tightlist}{%
  \setlength{\itemsep}{0pt}\setlength{\parskip}{0pt}}

\pagestyle{empty}

\begin{document}

{\LARGE Data Analysis Rubric}

\vspace*{18pt}

\noindent
\hspace*{24pt} Each data analysis will be graded using the following rubric.
  There are a total of 4 points. See syllabus for conversion to letter grades. \\
\hspace*{24pt} Getting all `Meets Expectations' yields a grade of $3$, a
  a B. Some categories have values for exceeding. I may award half-credit for
  work \\
\hspace*{24pt}  that falls between descriptions.

% Generally getting \\
% \hspace*{24pt}  all Meets Expectations should be a B/B+; only three categories have
% values for exceeding expectations.  \\
% \hspace*{24pt}  scores halfway between two categories.


\vspace*{18pt}

\renewcommand{\arraystretch}{1.8}
\begin{tabular}{p{4cm}||p{6cm}|p{6cm}|p{6cm}}
& Needs Improvement & Meets Expectations & Exceeds Expectations \\
\hline \hline
Thesis Statement
  & The thesis statement may be vague, overly general, or too specific. \textbf{(0)}
  & Clear conclusions are given that satisfy the requirements of the
  assignment and require statistical analysis in their argument. \textbf{(0.5)}
  & The thesis statement shows deep insight into the dataset by drawing
  specific non-trivial conclusions that require a very careful or
  multi-level analysis of the data. Results may be particularly
  surprising or interesting. \textbf{(0.75)} \\ \hline
Evidence
  & Aspects of the argument may incorrectly draw conclusions from the
  given data or may be tangential or irrelevant to making the author's
  point. \textbf{(0)}
  & Statistical evidence is seamlessly referenced and integrated into the
  paper's arguments. Arguments clearly flow from the thesis and use induction,
  deduction, or a combination thereof to make a clear case for the author's
  argument. \textbf{(0.5)}
  & The paper incorporates multiple independent lines of argument or
  successfully argues a particularly difficult thesis statement. Usually
  given only in tandem with an `Exceeds Expectations' thesis. \textbf{(1.0)}\\ \hline
Visualizations \& Models
  & Models, tables, and graphics may not be cohesively woven into the
  argument of the paper nor always appropriately applied. Graphs may
  make poor choices in terms of colors, data types, or fail to include
  proper labels. The command of some methods and theories under
  consideration may be weak or shaky. \textbf{(0)}
  & All models, tables and graphics are appropriately used and
  statistically sound. Graphics are properly labelled and
  visually pleasing. \textbf{(0.5)}
  & Inventive use of models or graphics. Graphics may contain many
  interwoven layers that increase information density without becoming
  too busy; models may include newly constructed variables or be fit on
  a different level of analysis than the raw data. \textbf{(0.75)} \\ \hline
Organization
  & Some aspects of the paper are not effectively integrated. There may be
  parts of the paper that do not further a distinct or coherent point. \textbf{(0)}
  & There is a logical structure appropriate to the subject. Sophisticated
  transitional sentences develop an idea from the previous one or identify
  their logical relations. The reader is guided through the chain of reasoning
  or progression of ideas. Graphics and tables are included at appropriate
  points. \textbf{(0.25)}
  & N/A \\ \hline
\end{tabular}

\renewcommand{\arraystretch}{1.8}
\begin{tabular}{p{4cm}||p{6cm}|p{6cm}|p{6cm}}
& Needs Improvement & Meets Expectations & Exceeds Expectations \\
\hline \hline
Style
  & Sentence structure tends to be repetitious; errors in usage and
  mechanics sometimes interfere with the writer's ability to communicate the
  purpose of the paper. The tone or intended audience of the piece may be
  inconsistent or not in line with the instructions. \textbf{(0)}
  & The author demonstrates a command of good writing style through a
  variety of sentence structures and word choices. Statistical results are
  woven into the narrative rather than distracting from it. \textbf{(0.25)}
  & N/A \\ \hline
Execution
  &  The writing contains errors and omissions that begin to impede on the
  author's ability to make overall arguments. \textbf{(0)}
  & The piece should is almost entirely free of spelling, punctuation, and
  grammatical errors. \textbf{(0.25)}
  & N/A \\ \hline
Directions
  & One or more directions in formatting or uploading the piece may not have been followed. \textbf{(0)}
  & All directions are followed and every aspect of the assignment answered in full. \textbf{(0.25)}
  & N/A \\ \hline
Presentation\textsuperscript{*}
  & The presentation may presuppose familiarity with the data at hand or fail
  to make a cohesive argument. The presenter shows a combination of lacking in
  preparation or understanding of the material. \textbf{(0)}
  & The presentation follows a logical structure, makes a compelling argument,
  and is interesting to and appropriate for the classroom audience. The presenter
  has clearly practiced the material and delivers their results confidently. \textbf{(0.5)}
  & N/A \\
\end{tabular}

\vspace*{68pt}

\textsuperscript{*} - If the data analysis has no presentation component, this
becomes class participation during the week(s) that we are working on them.

\end{document}

