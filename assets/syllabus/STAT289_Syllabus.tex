\documentclass[12pt]{article}

\usepackage{fontspec}
\usepackage{geometry}
\usepackage{lastpage}
\usepackage{fancyhdr}
\usepackage{hyperref}

\geometry{top=1in, bottom=1in, left=1in, right=1in, marginparsep=4pt, marginparwidth=1in}

\renewcommand{\headrulewidth}{0pt}
\pagestyle{fancyplain}
\fancyhf{}
\cfoot{\thepage\ of \pageref{LastPage}}

\setlength{\parindent}{0pt}
\setlength{\parskip}{0pt}

% \setromanfont [Ligatures={Common}, Numbers={OldStyle}, Variant=01,
%  BoldFont={LinLibertine_RB.otf},
%  ItalicFont={LinLibertine_RI.otf},
%  BoldItalicFont={LinLibertine_RBI.otf}
%  ]{LinLibertine_R.otf}

\usepackage{tikz}
\def\checkmark{\tikz\fill[scale=0.4](0,.35) -- (.25,0) -- (1,.7) -- (.25,.15) -- cycle;}

\usepackage{xunicode}
\defaultfontfeatures{Mapping=tex-text}

\setromanfont{YaleNew}

\begin{document}

\begin{center}
{\bf MATH 289: Applied Regression Analysis, Fall 2017} \\
Tuesday, Thursday 09:00-10:15 \quad JPSN, G28\\
\end{center}

\bigskip

\noindent
\begin{tabular}{ l l }
{\bf Instructor:} &  {\bf Taylor Arnold} \\
E-mail: & \href{mailto:tarnold2@richmond.edu}{tarnold2@richmond.edu} \\
Office: & Jepson Hall, Rm 218 \\
Office hours: & Tuesday, Thursday 10:30-12:00 or by appointment
\end{tabular}

\vspace{0.5cm}

\textbf{Computing:} \vspace{6pt}

The focus of this course will be on applied statistics and data
analysis over symbolic mathematics. To facilitate this, nearly
every class assignment and exam will involve some form of
computing. No prior programming experience is assumed or
required. \\

We will use the \textbf{R} programming environment throughout the
semester. It is freely available for all major operating systems and
is pre-installed on many campus computers. You can download it and
all supporting files for your own machine via these links:
\begin{center}
\url{https://cran.r-project.org/} \\
\url{https://www.rstudio.com/}
\end{center}
I strongly recommend using your own machine for the assignments
in this course. The lab computers will be available as well,
though I find them to be quite slow. We will devote a substantial
amount of the time learning how to work within the R programming
framework.

\vspace{0.4cm}

\textbf{Course Website:} \vspace{6pt}

All of the materials and assignments for the course will be posted
on the class website:
\begin{quote}
\url{https://statsmaths.github.io/stat289}
\end{quote}
At the end of the semester, this version of the course
will be archived and available for your reference.

\vspace{0.4cm}

\textbf{GitHub:} \vspace{6pt}

All of your work for this semester will be submitted through GitHub,
the same platform that hosts our website. You'll need to set up a free
account, which we will cover during the week of class.

\vspace{0.4cm}

\textbf{Labs:} \vspace{6pt}

Most class meetings, particularly in the first half of the semester,
will have an interactive lab associated with it. These consist of a
set of questions that must be answered with either small snippets of
code or short descriptive answers. Your solutions must be uploaded
to your GitHub page.

\vspace{0.4cm}

\textbf{Projects:} \vspace{6pt}

You will complete three data-oriented projects throughout
the semester. These are short written documents that mix code,
graphics, and prose to provide a comprehensive analysis of a
data set.  These must also be uploaded to GitHub.

\vspace{0.4cm}

\textbf{Exams:} \vspace{6pt}

This course has no exams, final or otherwise.

\vspace{0.4cm}

\textbf{Grades:} \vspace{6pt}

All grades in this course will be given on as a letter grade. \\

I expect most students to get full marks (A) for labs and participation.
Students found to be delinquent in either will first receive a written
warning, followed by an initial 50\% reduction (C) in the respective grade,
and finally a 100\% reduction (F). \\

Your final grade will be determined by converting all grades into
a numeric scale as follows (pluses increase the number by 0.33 and minuses
decrease the number by 0.33):

\begin{center}
\begin{tabular}{c || c}
Numeric Score & Final Grade \\
\hline \hline
4 & A  \\
3 & B  \\
2 & C  \\
1 & D  \\
0 & F
\end{tabular}
\end{center}

I want to make the grading extremely transparent, so your final grade will
simply consist of taking your weighted numerical average using the following
weights:

\begin{itemize}\setlength\itemsep{0em}
\item Labs, 25\%
\item Projects, 75\% (25\% each)
\end{itemize}

And reading off of the following chart (grades are rounded to the
second digit):

\begin{center}
\begin{tabular}{c || c}
Numeric Score & Final Grade \\
\hline \hline
3.84 - 4.00 & A  \\
3.50 - 3.83 & A- \\
3.17 - 3.49 & B+ \\
2.84 - 3.16 & B  \\
2.50 - 2.83 & B- \\
2.17 - 2.49 & C+ \\
1.84 - 2.16 & C  \\
1.50 - 1.83 & C- \\
0.00 - 1.49 & F
\end{tabular}
\end{center}

% \vspace{0.5cm}

% \textbf{Weekly Topics:} \vspace{6pt}

% These topics are subject to change based on the pace of the course, but
% give a good sense of roughly what we are going to cover:

% \vspace{0.5cm}

% \def\labelitemi{}
% \def\labelitemii{}

% \begin{itemize}\setlength\itemsep{0em}
% \item WEEK 01 - Introduction to R and RMarkdown
% \item WEEK 02 - Basic Graphics
% \item WEEK 03 - Variable Types
% \item WEEK 04 - Data Collection
% \item WEEK 05 - Data Manipulation
% \item WEEK 06 - Simple Linear Models
% \item WEEK 07 - Multivariate Linear Models
% \item WEEK 08 - Spatial Data
% \item WEEK 09 - Theories of Data Visualisation I
% \item WEEK 10 - Theories of Data Visualisation II
% \item WEEK 11 - Tidy Data
% \item WEEK 12 - Relational Data
% \item WEEK 13 - Strings and Dates
% \item WEEK 14 - Penalized Regression and Text Processing
% \end{itemize}

\end{document}





